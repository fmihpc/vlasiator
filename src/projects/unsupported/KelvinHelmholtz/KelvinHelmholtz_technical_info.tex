\documentclass[a4paper,10pt]{scrartcl}
\usepackage[utf8x]{inputenc}
\usepackage[left=1.5cm, right=1.5cm, top=1.5cm, bottom=2.5cm]{geometry}
\usepackage[autolanguage,np]{numprint}
\usepackage{tabularx}

\usepackage{hyperref}
\hyperbaseurl{.}

%opening
\title{
\Huge{Vlasiator test cases technical information} \\
\LARGE{KelvinHelmholtz}
}
\author{Yann Kempf}
\date{Updated on \today}

\begin{document}

\maketitle

\begin{abstract}
   This document gives technical information on the KelvinHelmholtz test case.
\end{abstract}

\section{Purpose}
Try to produce the Kelvin-Helmholtz instability in Vlasiator.


\section{Implementation}
Code originally copied from Riemann1, there is here an \verb=enum= for the \verb=TOP=/\verb=BOTTOM= states. One region centred on $z=0$ (\verb=TOP=) gets a distinct velocity and density state (\textit{e.\ g}.\ high velocity, low density), separated by a boundary which can be straight or have sinusoidal perturbations. The offset of the boundary from the $x$-axis is user-set.

\section{Options}
The options available in the \verb=cfg= file are:

\begin{tabularx}{\textwidth}{lX}
   \verb=rho[12]= & Number density (m\textsuperscript{-3}) \\
   \verb=T[12]= & Temperature (K) \\
   \verb=V[xyz][12]= & Velocity (m/s) \\
   \verb=B[xyz][12]= & Magnetic field (T) \\
   \verb=lambda= & Boundary perturbation wavelength (m) \\
   \verb=amp= & Boundary perturbation amplitude (m) \\
   \verb=offset= & Boundary offset from the $x$-axis (m) \\
   \verb=nSpaceSamples= & Number of sampling points along spatial dimensions within a spatial cell, includes the corners (minimum 2) \\
   \verb=nVelocitySamples= & Number of sampling points along velocity dimensions within a velocity cell, includes the corners (minimum 2)
\end{tabularx}




\end{document}