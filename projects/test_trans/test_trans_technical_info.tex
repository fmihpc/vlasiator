\documentclass[a4paper,10pt]{scrartcl}
\usepackage[utf8x]{inputenc}
\usepackage[left=1.5cm, right=1.5cm, top=1.5cm, bottom=2.5cm]{geometry}
\usepackage[autolanguage,np]{numprint}
\usepackage{tabularx}

\usepackage{hyperref}
\hyperbaseurl{.}

%opening
\title{
\Huge{Vlasiator test cases technical information} \\
\LARGE{test\_trans}
}
\author{Yann Kempf}
\date{Updated on \today}

\begin{document}

\maketitle

\begin{abstract}
   This document gives technical information on the test\_trans test case.
\end{abstract}

\section{Purpose}
Test the isotropy of the spatial part of the Vlasov solver. Can be used for isotropy and order of accuracy but becomes intensive with good grid spacings in three dimensions.


\section{Implementation}
Without field solver can be in periodic boundary conditions or not. Initialises symmetrically in each octant a cube with non-zero density, imposes a velocity along the diagonals of the space. One can then check visually or by calculating norms using VisIt or \verb=vlsvdiff_[SD]P=. The position of the initial cells can be set, the same cell position is used in space and velocity.

\section{Options}
The option available in the \verb=cfg= file is:

\begin{tabularx}{\textwidth}{lX}
   \verb=cellPosition= & Position of the centre of the cells initiated (same used in velocity and space, in cell lengths, \textit{i.\ e}.\ 1.5 corresponds to the second cell from the centre)
\end{tabularx}




\end{document}