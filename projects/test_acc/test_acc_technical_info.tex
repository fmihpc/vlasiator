\documentclass[a4paper,10pt]{scrartcl}
\usepackage[utf8x]{inputenc}
\usepackage[left=1.5cm, right=1.5cm, top=1.5cm, bottom=2.5cm]{geometry}
\usepackage[autolanguage,np]{numprint}
\usepackage{tabularx}

\usepackage{hyperref}
\hyperbaseurl{.}

%opening
\title{
\Huge{Vlasiator test cases technical information} \\
\LARGE{test\_acc}
}
\author{Yann Kempf}
\date{Updated on \today}

\begin{document}

\maketitle

\begin{abstract}
   This document gives technical information on the test\_acc test case.
\end{abstract}

\section{Purpose}
Test the isotropy of the acceleration part of the Vlasov solver. Can be used for isotropy and order of accuracy.

\section{Implementation}
Uses a single spatial cell in periodic boundaries without field solver. Initialises symmetrically in each octant a cube with non-zero density, imposes an artificial acceleration along the diagonals of the space. One can then check the result visually or by calculating norms using VisIt (done on my MSc thesis for instance). The position of the initial cells can be set so that one can also generate reference final states for comparison. The acceleration is artificial in that it is hand-coded into the \verb=calcAccFace[XYZ]= functions in the header file.

\section{Options}
The options available in the \verb=cfg= file are:

\begin{tabularx}{\textwidth}{lX}
   \verb=SPEED= & Artificially imposed acceleration value \\
   \verb=v_min= & Minimum value of the cubes initialised with non-zero density in velocity space \\
   \verb=v_max= & Maximum value of the cubes initialised with non-zero density in velocity space
\end{tabularx}




\end{document}