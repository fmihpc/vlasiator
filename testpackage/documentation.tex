\documentclass[a4paper,10pt]{article}
\usepackage[utf8]{inputenc}
\usepackage{url}
\usepackage{float}

%opening
\title{Testpackage documentation}

\begin{document}

\maketitle

\tableofcontents

\newpage

\section{Before starting}

Compile vlasiator with the testpackage flags:

\begin{verbatim}
make -j 8 testpackage
\end{verbatim}

\section{Using testpackage on Voima} \label{sec:voima}

To run the test package:

\begin{verbatim}
cd /lustre/my_folder/
cp -r ~/vlasiator/test_package .
cd test_package
ln -s ~/vlasiator/vlasiator_executable .
./small_test_voima.sh
\end{verbatim}


\section{Running testpackage on a local machine}

Set up the test package:

\begin{verbatim}
cp small_test_voima.sh small_test_new.sh
\end{verbatim}

Configure the sh files change the my\_test\_directory to something else:

\begin{verbatim}
sed -i 's/reference_dir=.*/reference_dir="\/my_test_directory\/small_tests"/g' small_test_new.sh
\end{verbatim}

(Change the revision number to the current revision number (optional))

\begin{verbatim}
sed -i 's/reference_revision=.*/reference_revision="0a8be8ac087c2da56556e4f56fcb3e5826aa6f38"/g' small_test_new.sh
\end{verbatim}


Run the test package once to create verification files which can be used to compare 
different branches with the current master version:

\begin{verbatim}
cd /lustre/my_folder/
cp -r ~/vlasiator/test_package .
cd test_package
ln -s ~/vlasiator/vlasiator_executable .
sed -i 's/create_verification_files=0/create_verification_files=1/g' small_test_new.sh
./small_test_new.sh
sed -i 's/create_verification_files=1/create_verification_files=0/g' small_test_new.sh
\end{verbatim}

\subsection{Running the testpackage on local machine}

Finally, similarly as in Section \ref{sec:voima}:

\begin{verbatim}
cd /lustre/my_folder/
cp -r ~/vlasiator/test_package .
cd test_package
ln -s ~/vlasiator/vlasiator_executable .
./small_test_new.sh
\end{verbatim}


\section{For more detailed information on the test package}

Directly from the README:

\begin{verbatim}
Test package consists of three parts:
   1. settings for different computers
   2. settings for different tests
   3. running the tests
for the first two parts the scripts for short and medium test are separate.

1. Computer settings
   - files are named: short/medium_test_hermit.sh, this is the file that is submitted
   - in these files user can define settings for running the program
   - define location for reference data directory
   - with variable create_verification_files you can define if you want to use the existing reference data or if you want first run the data for reference.
   - NOTE that if you choose to run the reference data first, the reference data directory must be somewhere from where you can submit runs

2. Test settings
   - files are named short/medium_test_definitions.sh
   - User can choose which tests are run. At the moment options are Fluctuations and Magnetosphere
   - Define the vlsv file for comparison
   - In arrays variables_name and variables_component use defines which parameters are compared.
   - variables_name has the name of the variable and variables_component have the number of the component that is compared
   - variables_name and variables_component must have same number of elements
   - For example:
      variables_name=( "rho" "rho_v" "rho_v" "rho_v" "B" "B" "B" "E" "E" "E" )
      variables_components=( 0 0 1 2 0 1 2 0 1 2 )
      -> e.g. 1st variable for comparison is variables_name[1] variables_component[1] = rho 0,
              4th is  variables_name[4] variables_component[4] = rho_v 4


3. Running tests
   - writes the results of the comparison in test.o file
\end{verbatim}


\end{document}
